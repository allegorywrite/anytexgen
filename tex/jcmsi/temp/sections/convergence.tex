\subsection{Convergence Analysis}
\label{sec:convergence}

This section analyzes the convergence of the proposed ASP-PGF algorithm. We establish that the ADMM-SVGD iterations converge to a KKT point of the constrained KL minimization problem. The analysis adapts standard ADMM convergence arguments to our setting with SVGD updates on manifolds.

We first state the main convergence theorem, followed by key propositions and lemmas.

\begin{theorem}[Convergence of ASP-PGF]
\label{thm:convergence}
Let $\{ \{q_i^k\}_{i \in {\mathcal{V}}}, \{\zeta_{ij}^k\}_{(i,j) \in {\mathcal{E}}}, \{\lambda_{ij}^k\}_{(i,j) \in {\mathcal{E}}} \}_{k \ge 0}$ be the sequence generated by the ADMM-SVGD iteration described in Section \ref{subsec:admm_updates} (\Eqref{eq:admm_zeta_update}, \Eqref{eq:admm_svgd_update}, \Eqref{eq:admm_lambda_update_kl}). Assume the following conditions hold:
\begin{itemize}
    \item[\textbf{A1}] (Lipschitz gradients) For every edge $(i,j) \in {\mathcal{E}}$, the cost function $c_{ij}(X_i, X_j) = \frac{1}{2} \| \log({\tilde{Z}}_{ij}^{-1} X_i^{-1} X_j)^{\vee} \|^2_{\Omega_{ij}}$ has a Lipschitz-continuous gradient with respect to $X_i$ and $X_j$ on SE(3). Let $L_{ij}$ be the Lipschitz constant.
    \item[\textbf{A2}] (Bounded entropy/particles) Each particle set $\{X_{i,l}\}_{l=1}^m$ representing $q_i^k$ remains within a compact subset of SE(3) for all $k$, ensuring the differential entropy ${\mathcal{H}}[q_i^k]$ is finite and bounded.
    \item[\textbf{A3}] (Penalty parameter) The penalty parameter $\rho > 0$ is chosen sufficiently large, specifically $\rho > \max_{(i,j)} L_{ij}$.
    \item[\textbf{A4}] (Initial feasibility) The initial dual variables satisfy $\sum_{(i,j) \in {\mathcal{E}}} \|\lambda_{ij}^0\|^2 < \infty$.
\end{itemize}
Define the merit function (augmented Lagrangian plus proximal terms):
\begin{equation}
\begin{aligned}
\Phi^k = {\mathcal{L}}_{\rho}(\{q_i^k\}, \{\zeta_{ij}^k\}, \{\lambda_{ij}^k\}) + \frac{\gamma}{2} \sum_{i \in {\mathcal{V}}} \|q_i^k - q_i^{k-1}\|^2_{\text{W}_2}, \quad \gamma := \frac{4}{\rho},
\label{eq:merit_function}
\end{aligned}
\end{equation}
where $\| \cdot \|_{\text{W}_2}$ denotes a suitable distance metric between distributions (e.g., Wasserstein-2 distance, though the specific metric depends on the SVGD analysis details). Then the following hold:
\begin{enumerate}
    \item \textbf{(Boundedness)} The sequence $\{\Phi^k\}$ is bounded below by the infimum of the variational free energy $F[q] = \mathbb{E}_q[C({\mathbf{X}})] - {\mathcal{H}}[q]$.
    \item \textbf{(Monotone Descent)} The merit function is non-increasing, satisfying $\Phi^{k+1} \le \Phi^k - \kappa \Xi^k$ for some $\kappa > 0$, where $\Xi^k$ measures the progress at iteration $k$:
    \begin{equation}
    \begin{aligned}
    \Xi^k = \sum_{i \in {\mathcal{V}}} \|q_i^{k+1} - q_i^k\|^2_{\text{W}_2} + \sum_{(i,j) \in {\mathcal{E}}} (\zeta_{ij}^{k+1} - \mathbb{E}_{q_i^{k+1}, q_j^{k+1}}[c_{ij}])^2.
    \label{eq:progress_measure}
    \end{aligned}
    \end{equation}
    \item \textbf{(Primal Feasibility)} The constraint violation converges to zero: $\lim_{k \to \infty} |\zeta_{ij}^k - \mathbb{E}_{q_i^k, q_j^k}[c_{ij}]| = 0$ for all $(i,j) \in {\mathcal{E}}$.
    \item \textbf{(Stationarity)} Any accumulation point $(\{q_i^*\}, \{\zeta_{ij}^*\}, \{\lambda_{ij}^*\})$ of the sequence satisfies the Karush-Kuhn-Tucker (KKT) conditions for the constrained KL minimization problem \Eqref{eq:kl_min_constrained}. In particular, each $q_i^*$ is a stationary distribution for the SVGD update driven by the effective potential derived from the ADMM step, satisfying:
    \begin{equation}
    \begin{aligned}
    \mathbb{E}_{q_i^*} \left[ \sum_{j \in {\mathcal{N}}_i} W_{ij}^* \, \nabla_{X_i} \mathbb{E}_{q_j^*}[c_{ij}(X_i, X_j)] \cdot \boldsymbol{\phi}(X_i) + \nabla_{X_i} \cdot \boldsymbol{\phi}(X_i) \right] = 0,
    \label{eq:kkt_stationarity}
    \end{aligned}
    \end{equation}
    for all test functions $\boldsymbol{\phi}$ in the RKHS, where $W_{ij}^* = \lambda_{ij}^* + \rho(\zeta_{ij}^* - \mathbb{E}_{q_i^*, q_j^*}[c_{ij}])$.
    \item \textbf{(Iteration Complexity)} For any $\epsilon > 0$, the number of iterations $K(\epsilon)$ required to reach an $\epsilon$-stationary point (in the sense that $\Xi^k \le \epsilon^2$) is bounded by:
    \begin{equation}
    \begin{aligned}
    K(\epsilon) \le \left\lceil \frac{\Phi^0 - F^*}{\kappa \epsilon^2} \right\rceil,
    \label{eq:iteration_complexity}
    \end{aligned}
    \end{equation}
    where $F^*$ is the infimum of the free energy.
\end{enumerate}
\end{theorem}

The proof of Theorem \ref{thm:convergence} relies on the following propositions and lemmas.

\begin{proposition}[Properties of the Subproblems]
\label{prop:subproblems}
Under assumptions A1-A3:
\begin{enumerate}
    \item[\textbf{(P1)}] For fixed $q_i^k, q_j^k, \lambda_{ij}^k$, the $\zeta$-update subproblem \Eqref{eq:admm_zeta_update_min} is strongly convex in $\zeta_{ij}$ with a unique solution given by \Eqref{eq:admm_zeta_update}.
    \item[\textbf{(P2)}] For fixed $\zeta_{ij}^{k+1}, \lambda_{ij}^k$, the objective function ${\mathcal{L}}_{\rho, i}$ in the $q$-update subproblem \Eqref{eq:admm_q_update_min} is such that the target distribution $q_i^*$ defined via \Eqref{eq:admm_q_solution_form} has a Lipschitz continuous log-gradient due to A1. This ensures the SVGD update is well-defined.
    \item[\textbf{(P3)}] The entropy term ${\mathcal{H}}[q_i]$ acts as a barrier. Combined with A2 (compact support for particles), the feasible set of distributions $q_i$ with bounded free energy is effectively compact in a suitable topology.
\end{enumerate}
\end{proposition}

\begin{proof}[Proof of Proposition \ref{prop:subproblems}]
\textbf{(P1) Strong convexity of the $\zeta$-subproblem:}
The $\zeta$-update \Eqref{eq:admm_zeta_update} minimizes $f(\zeta_{ij}) = \zeta_{ij} + \lambda_{ij}^k (\zeta_{ij} - \mathbb{E}_{q_i^k, q_j^k}[c_{ij}]) + \frac{\rho}{2} (\zeta_{ij} - \mathbb{E}_{q_i^k, q_j^k}[c_{ij}])^2$. This is quadratic in $\zeta_{ij}$ with $\frac{\partial^2 f}{\partial \zeta_{ij}^2} = \rho > 0$ (by A3), ensuring $\rho$-strong convexity and a unique minimizer.

\textbf{(P2) Smoothness of ${\mathcal{L}}_{\rho, i}$ for $q$-update:}
The relevant part of the augmented Lagrangian for $q_i$ is ${\mathcal{L}}_{\rho, i}(q_i) = -{\mathcal{H}}[q_i] - \sum_{j \in {\mathcal{N}}_i} [ \lambda_{ij}^k \mathbb{E}_{q_i, q_j^k}[c_{ij}] + \frac{\rho}{2} (\zeta_{ij}^{k+1} - \mathbb{E}_{q_i, q_j^k}[c_{ij}])^2 ]$.
Assumption A1 (Lipschitz gradient of $c_{ij}$) implies $\nabla_{X_i} \mathbb{E}_{q_j^k}[c_{ij}(X_i, X_j)]$ is Lipschitz in $X_i$. Under standard SVGD assumptions and A2 (compact support), the functional gradients $\nabla_{q_i} \mathbb{E}_{q_i, q_j^k}[c_{ij}]$ and $\nabla_{q_i} {\mathcal{H}}[q_i]$ are Lipschitz w.r.t. $q_i$. Thus, $\nabla_{q_i} {\mathcal{L}}_{\rho, i}$ is Lipschitz, meaning ${\mathcal{L}}_{\rho, i}$ is $\beta$-smooth. This ensures the target distribution for SVGD in \Eqref{eq:admm_q_solution_form} has a Lipschitz continuous log-gradient.

\textbf{(P3) Lower boundedness and compact sublevel sets:}
The negative entropy $-{\mathcal{H}}[q_i]$ in the variational free energy $F[q]$ (\Eqref{eq:kl_min_objective}) acts as a barrier, ensuring $F[q]$ is bounded below (as $c_{ij}$ is bounded below). Sublevel sets $\{q \mid F[q] \le C\}$ require bounded ${\mathcal{H}}[q_i]$, restricting distribution spread. Combined with A2 (compact particle support), relevant distributions $\{q_i\}$ lie in a compact set, ensuring iterates remain bounded.
\end{proof}

\begin{lemma}[Bounded Lagrange Sequence]
\label{lem:bounded_lambda}
Under the assumptions, the sequence of dual variables $\{\lambda_{ij}^k\}$ is bounded. Specifically, the change in dual variables satisfies:
\begin{equation}
\begin{aligned}
\|\lambda^{k+1} - \lambda^k\|^2 \le C_1 \sum_{i \in {\mathcal{V}}} \|q_i^{k+1} - q_i^k\|^2_{\text{W}_2} + C_2 \sum_{(i,j) \in {\mathcal{E}}} (\zeta_{ij}^{k+1} - \mathbb{E}_{q_i^{k+1}, q_j^{k+1}}[c_{ij}])^2,
\label{eq:lambda_diff_bound}
\end{aligned}
\end{equation}
for some constants $C_1, C_2 > 0$ depending on $\rho$ and the Lipschitz constants. This implies $\sum_{k=0}^\infty \|\lambda^{k+1} - \lambda^k\|^2 < \infty$.
\end{lemma}

\begin{proof}[Proof of Lemma \ref{lem:bounded_lambda}]
The $\lambda$-update rule \Eqref{eq:admm_lambda_update_kl} is $\lambda_{ij}^{k+1} - \lambda_{ij}^{k} = \rho r_{ij}^k$, where $r_{ij}^k := \zeta_{ij}^{k+1} - \mathbb{E}_{q_i^{k+1}, q_j^{k+1}}[c_{ij}]$ is the primal residual.
Squaring and summing gives $\|\lambda^{k+1} - \lambda^k\|^2 = \rho^2 \sum_{(i,j) \in {\mathcal{E}}} (r_{ij}^k)^2$.
The progress measure $\Xi^k = \sum_{i \in {\mathcal{V}}} \|q_i^{k+1} - q_i^k\|^2_{\text{W}_2} + \sum_{(i,j) \in {\mathcal{E}}} (r_{ij}^k)^2$ (\Eqref{eq:progress_measure}).
From Lemma \ref{lem:merit_decrease}, $\sum_{k=0}^\infty \kappa \Xi^k < \infty$, implying $\sum_{k=0}^\infty \Xi^k$ converges.
Since $\Xi^k \ge 0$, its convergence implies $\sum_{k=0}^\infty \sum_{(i,j) \in {\mathcal{E}}} (r_{ij}^k)^2 < \infty$.
Therefore, $\sum_{k=0}^\infty \|\lambda^{k+1} - \lambda^k\|^2 = \rho^2 \sum_{k=0}^\infty \sum_{(i,j) \in {\mathcal{E}}} (r_{ij}^k)^2 < \infty$.
This means $\|\lambda^{k+1} - \lambda^k\| \to 0$. The inequality \Eqref{eq:lambda_diff_bound} follows with $C_1 \ge 0, C_2 = \rho^2$, as $\|\lambda^{k+1} - \lambda^k\|^2 = \rho^2 \sum (r_{ij}^k)^2 \le C_1 \sum \|q_i^{k+1} - q_i^k\|^2_{\text{W}_2} + \rho^2 \sum (r_{ij}^k)^2$.
The convergence of $\sum \|\lambda^{k+1} - \lambda^k\|^2$ is key, indicating $\{\lambda^k\}$ is a Cauchy sequence in its differences, leading to boundedness or convergence.
\end{proof}

\begin{lemma}[Sufficient Decrease of Merit Function]
\label{lem:merit_decrease}
Under the assumptions, the merit function $\Phi^k$ decreases sufficiently at each iteration:
\begin{equation}
\begin{aligned}
\Phi^k - \Phi^{k+1} \ge \frac{\rho}{2} \sum_{(i,j) \in {\mathcal{E}}} (\zeta_{ij}^{k+1} - \mathbb{E}_{q_i^{k+1}, q_j^{k+1}}[c_{ij}])^2 + \frac{\gamma}{2} \sum_{i \in {\mathcal{V}}} \|q_i^{k+1} - q_i^k\|^2_{\text{W}_2}.
\label{eq:merit_decrease_bound}
\end{aligned}
\end{equation}
This inequality establishes the relationship $\Phi^k - \Phi^{k+1} \ge \kappa \Xi^k$ stated in Theorem \ref{thm:convergence}(2).
\end{lemma}

\begin{proof}[Proof of Lemma \ref{lem:merit_decrease}]
The proof combines descent from $\zeta$-update, $q$-update (SVGD), and the $\lambda$-update.
\textbf{1. $\zeta$-update:} From Prop. \ref{prop:subproblems}(P1) ($\rho$-strong convexity), standard ADMM analysis (cf. \cite{Boyd2011}) shows ${\mathcal{L}}_{\rho}(\{q^k\}, \{\zeta^k\}, \{\lambda^k\}) - {\mathcal{L}}_{\rho}(\{q^k\}, \{\zeta^{k+1}\}, \{\lambda^k\}) \ge \frac{\rho}{2} \sum_{(i,j)} (\zeta_{ij}^{k+1} - \mathbb{E}_{q_i^k, q_j^k}[c_{ij}])^2$. (The exact form depends on relating to $\|\zeta^{k+1}-\zeta^k\|^2$).

\textbf{2. $q$-update (SVGD):} The SVGD step for $q_i$ approximates minimizing ${\mathcal{L}}_{\rho, i}(q_i)$ (\Eqref{eq:admm_q_update_min}). Interpreting SVGD as a gradient flow (e.g., KL/Wasserstein geometry), it exhibits sufficient decrease. Assuming SVGD acts like a proximal update:
${\mathcal{L}}_{\rho}(\{q^k\}, \{\zeta^{k+1}\}, \{\lambda^k\}) - {\mathcal{L}}_{\rho}(\{q^{k+1}\}, \{\zeta^{k+1}\}, \{\lambda^k\}) \ge \frac{\gamma}{2} \sum_{i} \|q_i^{k+1} - q_i^k\|^2_{\text{W}_2}$.
(Rigorous justification involves SVGD theory on manifolds for ${\mathcal{L}}_{\rho, i}$).

\textbf{3. Combining and $\lambda$-update:} The change in ${\mathcal{L}}_{\rho}$ from the $\lambda$-update is $\Delta {\mathcal{L}}_{\rho}^{\lambda} = \rho \sum_{(i,j)} (r_{ij}^k)^2$, where $r_{ij}^k = \zeta_{ij}^{k+1} - \mathbb{E}_{q^{k+1}}[c_{ij}]$.
Combining these (details in standard ADMM proofs, e.g., \cite{Boyd2011}) yields:
$\Phi^k - \Phi^{k+1} \ge \frac{\rho}{2} \sum_{(i,j)} (r_{ij}^k)^2 + \frac{\gamma}{2} \sum_{i} \|q_i^{k+1} - q_i^k\|^2_{\text{W}_2}$.
This is \Eqref{eq:merit_decrease_bound}. With $\gamma = 4/\rho$, $\kappa = \min(\rho/2, \gamma/2) = \min(\rho/2, 2/\rho)$.
\end{proof}
Now we are ready to prove Theorem  4.1.
\begin{proof}[Proof of Theorem \ref{thm:convergence}]
The proof leverages Prop. \ref{prop:subproblems} and Lemmas \ref{lem:bounded_lambda}, \ref{lem:merit_decrease}.

\textbf{1. (Boundedness):} Prop. \ref{prop:subproblems}(P3) states $F[q]$ is bounded below. Lemma \ref{lem:merit_decrease} shows $\Phi^k$ is non-increasing. Since $\Phi^k = {\mathcal{L}}_{\rho}^k + (\text{non-negative term})$, and ${\mathcal{L}}_{\rho}^k$ relates to $F[q]$, $\Phi^k$ is bounded below and thus converges.

\textbf{2. (Monotone Descent & Residual Convergence):} Lemma \ref{lem:merit_decrease} gives $\Phi^k - \Phi^{k+1} \ge \kappa \Xi^k$ with $\kappa > 0$. Summing yields $\sum_{k=0}^{K-1} \kappa \Xi^k \le \Phi^0 - \Phi^K$. As $K \to \infty$, $\Phi^0 - \Phi^K \to \Phi^0 - \Phi^* < \infty$. Thus, $\sum \kappa \Xi^k$ converges, implying $\sum \Xi^k$ converges. This requires $\lim_{k \to \infty} \Xi^k = 0$. From \Eqref{eq:progress_measure}, this means $\lim_{k \to \infty} \|q_i^{k+1} - q_i^k\|_{\text{W}_2} = 0$ and $\lim_{k \to \infty} (\zeta_{ij}^{k+1} - \mathbb{E}_{q_i^{k+1}, q_j^{k+1}}[c_{ij}]) = 0$.

\textbf{3. (Primal Feasibility):} The second limit from (2) is precisely primal feasibility: $\zeta_{ij}^* = \mathbb{E}_{q_i^*, q_j^*}[c_{ij}]$ at any accumulation point.

\textbf{4. (Stationarity):} Let $(\{q_i^*\}, \{\zeta_{ij}^*\}, \{\lambda_{ij}^*\})$ be an accumulation point. The limits from (2) and Lemma \ref{lem:bounded_lambda} ($\|\lambda^{k+1}-\lambda^k\| \to 0$) imply it's a fixed point.
Primal feasibility holds. The $\lambda$-update \Eqref{eq:admm_lambda_update_kl} becomes $0=0$. The $\zeta$-update optimality ($1 + \lambda_{ij}^* + \rho (\zeta_{ij}^* - \mathbb{E}_{q_i^*, q_j^*}[c_{ij}]) = 0$) simplifies to $1 + \lambda_{ij}^* = 0$.
The $q$-update \Eqref{eq:admm_svgd_update} implies $q_i^*$ is a stationary distribution for ${\mathcal{L}}_{\rho, i}$, leading to the KKT condition \Eqref{eq:kkt_stationarity} with $W_{ij}^* = \lambda_{ij}^*$. Thus, KKT conditions for \Eqref{eq:kl_min_constrained} are met.

\textbf{5. (Iteration Complexity):} From (2), $\sum_{k=0}^{K-1} \Xi^k \le \frac{1}{\kappa}(\Phi^0 - \Phi^K)$. Since $\Phi^K \ge F^*$ (infimum of free energy), $\sum_{k=0}^{K-1} \Xi^k \le \frac{1}{\kappa}(\Phi^0 - F^*)$.
This implies $\min_{0 \le k < K} \Xi^k \le \frac{1}{K} \sum_{k=0}^{K-1} \Xi^k \le \frac{\Phi^0 - F^*}{\kappa K}$.
To ensure $\min \Xi^k \le \epsilon^2$, we need $K \ge \frac{\Phi^0 - F^*}{\kappa \epsilon^2}$. Thus, $K(\epsilon) \le \lceil \frac{\Phi^0 - F^*}{\kappa \epsilon^2} \rceil$, an $\mathcal{O}(1/\epsilon^2)$ complexity.
\end{proof}
\vspace{2mm}

\textit{Remarks.} The analysis confirms ASP-PGF converges to a KKT point of the constrained KL minimization under standard assumptions. This justifies ADMM with SVGD for distributed probabilistic PGF, handling multi-modal distributions. Practical convergence speed depends on $\rho$ and the SVGD step size.
